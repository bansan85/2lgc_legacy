\documentclass{article}

\usepackage[utf8]{inputenc} %Pour les accents puisque ce fichier est en format UTF-8
\usepackage[T1]{fontenc}
\usepackage[frenchb]{babel} % Pour la césure je crois
\usepackage{geometry} %Pour geometry
\usepackage[hyperindex=true, colorlinks=true]{hyperref} %Pour /url pour un affichage correcte en PDF
\usepackage{bookmark} % Pour éviter le message d'erreur "Rerun to get outlines right or use package `bookmark'
\usepackage{amsmath} % Pour les maths "avancés"

\geometry
{
    a4paper,
    body={160mm,247mm}, % 21*29.7-2.5*2
    left=25mm,top=25mm,
    headheight=7mm,headsep=4mm,
    marginparsep=4mm,
    marginparwidth=27mm
}

\begin{document}
\author{LE GARREC Vincent, 2LGC, FRANCE}
\title{Guide du développeur de la librairie libeurocodes, un outil de calcul de structure conformément aux Eurocodes}
\maketitle
\newpage
\section{Présentation}
La librairie libeurocodes, programmée en C, est un ensemble de fonctions permettant de réaliser des calculs de structures conformément aux Eurocodes, intégrant également un module de calculs aux éléments finis. L'objectif est de réaliser une bibliothèque légère et performante de modélisation d'une structure permettant ainsi le dimensionnement et la vérification d'ouvrages ou d'éléments en matériau béton, métal, bois, mixte (béton-acier), maçonnerie et aluminimum vis-à-vis des charges d'exploitation, climatique (vent, neige) ou encore accidentelles (chocs, sismiques).
\par
La librairie est distribuée sous licence GPL, ce qui signifie que seules des applications adoptant pour la licence GPL peuvent utiliser la présente bibliothèque. Veuillez contacter l'auteur du projet si vous souhaitez obtenir la librairie sous une autre licence.
\par
\par
Les types, définis par {\texttt{typedef struct}} permettent de stocker toutes les informations du projet en mémoire telles que les noeuds, les barres ou encore les résultats des calculs et les fonctions permettent de gérer et d'ajouter des données au projet ou encore de réaliser les calculs pour en afficher les résultats.
\par
Les fonctions sont classées par norme. Ainsi, toute fonction se basant par exemple sur l'Eurocode~2-1-1 commencera par la dénomination {\texttt{\_1992\_1\_1}}. Ensuite, toujours sur le même principe, deux groupes de fonctions existent en plus sous la dénomination {\texttt{EF}}, qui contient tous les calculs aux éléments finis en exploitant les librairies {\texttt{cholmod}}\footnote{Sparse Cholesky factorization and update/downdate library : \url{http://www.cise.ufl.edu/research/sparse/cholmod}} et {\texttt{spqr}}\footnote{Multithreaded multifrontal sparse QR factorization library : \url{http://www.cise.ufl.edu/research/sparse/SPQR}}, et {\texttt{common}}, qui répertorie toutes les fonctions communes à toutes les normes (la gestion des allocations mémoire, le stockage des données de tout un projet et plus généralement toutes les fonctions divers et variées ne pouvant appartenir à aucune autre rubriques.)
\newpage
\section{Programme de base}
L'objectif de ce paragraphe est de présenter les différentes fonctions de la librairie "libeurocodes" dans un contexte didactique. L'ensemble des fonctions sera ensuite présenté de manière plus détaillée dans le chapitre suivant notamment en présentant de façon précise les algorithmes utilisés. Il est possible, par exemple, d'utiliser les fonctions suivantes :
\par
\begin{enumerate}
    \item {\texttt{projet\_init}} : doit être appelé en premier dans tous les cas.
    \item {\texttt{EF\_appuis\_ajout}} : défini un type d'appui (rotule, appui simple, encastrement, ...).
    \item {\texttt{EF\_noeuds\_ajout}} : ajoute les noeuds de la structure et attribut, si nécessaire, un type d'appui comme défini précédemment.
    \item {\texttt{\_1992\_1\_1\_sections\_ajout\_rectangulaire}} : défini une section rectangulaire en béton conformément à l'Eurocode 2-1-1. Il est également possible de définir des sections en T, carrée ou circulaire.
    \item {\texttt{\_1992\_1\_1\_materiaux\_ajout}} : défini le matériau béton en fonction de sa résistance caractéristique à la compression à 28 jours ($f_{ck}$).
    \item {\texttt{EF\_relachement\_ajout}} : défini un type de relâchement. Il est possible de relâcher uniquement les rotations.
    \item {\texttt{\_1992\_1\_1\_elements\_ajout}} : ajout un élément en béton (poteau, poutre, ...) à partir de toutes les données précédentes (section, matériau, noeuds, relachement). Il est également possible de discrétiser l'élément un nombre de tronçons de même longueur à définir.
    \item {\texttt{\_1990\_action\_ajout}} : ajoute un type d'action. Chaque type d'action peut contenir plusieurs charges de projet de même nature.
    \item {\texttt{\_1990\_groupe\_niveau\_ajout}} : défini différents niveaux de groupe. Les groupes de niveau 0 contien\-nent des actions, les groupes de niveau 1 contien\-nent des groupes de niveau 0, ...
    \item {\texttt{\_1990\_groupe\_ajout}} : ajoute un groupe d'actions ou un groupe de groupes à l'intérieur d'un niveau défini par la fonction {\texttt{\_1990\_groupe\_niveau\_ajout}}. Les éléments à l'intérieur d'un même groupe peuvent être combinés par une opérande de type AND, OR ou XOR.
    \item {\texttt{\_1990\_groupe\_ajout\_element}} : ajoute des éléments à l'intérieur des groupes définis par la fonction {\texttt{\_1990\_groupe\_ajout}}.
    \item {\texttt{\_1990\_action\_ajout\_charge\_ponctuelle\_noeud}} : ajoute une charge ponctuelle s'appliquant à un noeud. Il est également possible d'appliquer une charge ponctuelle à l'intérieur d'une barre.
    \item {\texttt{EF\_calculs\_initialise}} : initialise les calculs aux éléments finis.
    \item {\texttt{\_1992\_1\_1\_elements\_rigidite\_ajout\_tout}} : ajoute à la matrice de rigidité globale la rigidité de l'ensemble des éléments en béton.
    \item {\texttt{EF\_calculs\_genere\_sparse}} : calcule l'inverse de la matrice de rigidité globale partielle. Le terme partiel indique que la matrice de rigidité ne contient que les lignes et colonnes dont le déplacement est inconnu.
    \item {\texttt{EF\_calculs\_resoud\_charge}} : détermine à partir de la matrice de rigidité calculée précédemment les déplacements des noeuds, les efforts aux noeuds et les fonctions décrivant les sollicitations ($N$, $T_y$, $T_z$, $M_x$, $M_y$ et $M_z$) et les déplacements ($\theta{}_x$, $\theta{}_y$, $\theta{}_z$, $f_x$, $f_y$ et $f_z$).
    \item {\texttt{EF\_calculs\_affiche\_resultats}} : affiche les résultats d'une charge pour toutes les barres dans la console.
    \item {\texttt{projet\_free}} : libère tous les résultats et les zones mémoires allouées pour les calculs.
\end{enumerate}
\section{Exemple d'application en C}
\begin{verbatim}
TODO : INSERER LE CODE SOURCE D'UNE APPLICATION DE BASE!!!
\end{verbatim}
Et expliquer ici succintement le but de l'application.
\newpage
\section{Informations générales}
Les explications données dans le présent paragraphe sont valables pour toute la librarie quelque soit le module.
\subsection{Règles de rédaction du code}
L'usage de la tabulation est prohibé, elle doit être remplacée par 4 espaces afin que le code source soit lisible correctement à partir de l'importe quel ordinateur.\par
La vérification pour chaque allocation mémoire que ce soit pour \texttt{malloc} (stdlib.h), \texttt{list\_init} (list.h) ou les fonctions cholmod (cholmod.h) tel que \texttt{cholmod\-\_l\-\_allocate\-\_triplet}, \texttt{cholmod\-\_l\-\_triplet\-\_to\-\_sparse} et autres.\par
Rédigez la documentation en format latex par l'usage des commentaires //. L'ensemble des commentaires sera inséré dans le fichier latex et sera entouré des instructions \texttt{\\begin{verbatim}} et \texttt{\\end{verbatim}}.
Ainsi, pour insérer des formules, il suffit d'encadrer la formule par \texttt{\\end\{verbatim\}} \texttt{\\begin\{align*\}} et de finir par \texttt{\\end\{align*\}} \texttt{\\begin\{verbatim\}}.
Afin de rester sur la largeur de la page ne pas dépasser 95 caractères sur une même ligne en format verbatim. L'objectif de toujours insérer la documentation directement dans le code source permet d'avoir systématiquement une documentation à jour ainsi qu'un code source très clair.
Les commentaires spécifiques au code source sont rédigés par les commentaires \verb+\*+. Pensez aussi à vérifier l'orthographe.\par
Les malloc doivent être typé, cela afin d'être sûr d'allouer le bon type. Exemple : \begin{center}\texttt{(int*)malloc(6*sizeof(int));}\end{center}.
Les commentaires décrivant les fonctions se trouve dans le fichier source (\texttt{.c}) juste en dessous de la déclaration de l'entête. Exemple :
\begin{verbatim}
/* Description : Divise un tronçon en deux à la position coupure.
* Paramètres : Fonction* fonction : la variable contenant la fonction
*            : double coupure : position de la coupure
* Valeur renvoyée :
*   Succès : 0
*   Échec : -1 en cas de paramètres invalides :
*             (fonction == NULL) ou
*             (fonction->nb_troncons == 0)
*           -2 en cas d'erreur d'allocation mémoire
*/
\end{verbatim}
Il convient de toujours tester au début de la fonction la validité des paramètres, ceci principalement pour faciliter le debogage de l'application.
Pour vérifier si deux nombres flottants sont identiques, il convient d'utiliser \texttt{ERREUR\_RELATIVE\_EGALE(nb1, nb2)}.
En cas de vérification où l'échec est globalement critique pour l'application, il convient d'utiliser les fonctions \texttt{BUG} et \texttt{BUGTEXTE}. Voir la section \texttt{common\_erreurs} pour les détails.
\section{Module {\texttt{common}}}
\subsection{Fichier {\texttt{common\_erreurs}}}
Ce fichier permet de traiter les erreurs de l'application qui peuvent survenir pendant une utilisation normale ou anormale de l'application. Elles permettent de détecter plus facilement les erreurs et de remonter l'erreur jusqu'aux plus hautes instances du programme tout en assurant sa stabilité.
\subsubsection{Macros}
\input{common_erreurs_BUG_func.tex.tmp}
\input{common_erreurs_BUGTEXTE_func.tex.tmp}
\subsubsection{Structures de données}
\subsection{Fichier {\texttt{common\_fonction}}}
Ce fichier permet de gérer des fonctions méthématiques plus ou moins complexes. Le principal intérêt est de pouvoir travailler avec une fonction différentes pour chaque tronçon défini par l'utilisateur. Par exemple avoir la fonction $x^2+3*x$ pour x variant de -1.5 à 1.5 et la fonction $-2*x+5$ pour x variant de 1.5 à 2.0.
\subsubsection{Macros}
Aucune.
\subsubsection{Structures de données}
\input{common_fonction_Troncon_struct.tex.tmp}
\input{common_fonction_Fonction_struct.tex.tmp}
\subsubsection{Fonctions}
\noindent\hrulefill
\input{common_fonction_init_func.tex.tmp}
\input{common_fonction_init_com.tex.tmp}
\input{common_fonction_scinde_troncon_func.tex.tmp}
\input{common_fonction_scinde_troncon_com.tex.tmp}
\input{common_fonction_ajout_func.tex.tmp}
\input{common_fonction_ajout_com.tex.tmp}
\input{common_fonction_affiche_func.tex.tmp}
\input{common_fonction_affiche_com.tex.tmp}
\input{common_fonction_free_func.tex.tmp}
\input{common_fonction_free_com.tex.tmp}
\subsection{Fichier {\texttt{common\_maths}}}
Ce fichier permet de gérer les calculs avec prise en compte des imprécisions provenant de l'utilisation des nombres flottants.
\subsubsection{Macros}
\input{common_maths_ERREUR_RELATIVE_EGALE_func.tex.tmp}
\subsubsection{Structures de données}
Aucune.
\subsubsection{Fonctions}
\noindent\hrulefill
\input{common_math_arrondi_nombre_func.tex.tmp}
\input{common_math_arrondi_nombre_com.tex.tmp}
\input{common_math_arrondi_triplet_func.tex.tmp}
\input{common_math_arrondi_triplet_com.tex.tmp}
\input{common_math_arrondi_sparse_func.tex.tmp}
\input{common_math_arrondi_sparse_com.tex.tmp}
\subsection{Fichier {\texttt{common\_projet}}}
Ce fichier permet de gérer l'initialisation et la libération de la totalité de la mémoire pour un projet entier. Il contient également la structure de toutes les informations nécessaires à la modélisation et au stockage des résultats de calculs.
\subsubsection{Macros}
Aucune.
\subsubsection{Structures de données}
\input{common_projet_Type_Pays_struct.tex.tmp}
\input{common_projet_CombinaisonsEL_struct.tex.tmp}
\input{common_projet_List_Gtk_struct.tex.tmp}
\input{common_projet_Type_Element_struct.tex.tmp}
\input{common_projet_Beton_Donnees_struct.tex.tmp}
\input{common_projet_EF_struct.tex.tmp}
\input{common_projet_Projet_struct.tex.tmp}
\subsubsection{Fonctions}
\noindent\hrulefill
\input{common_projet_projet_init_func.tex.tmp}
\input{common_projet_projet_init_com.tex.tmp}
\input{common_projet_projet_free_func.tex.tmp}
\input{common_projet_projet_free_com.tex.tmp}
\subsection{Fichier {\texttt{common\_text}}}
Ce fichier contient la liste des messages à afficher à l'utilisateur.
\subsubsection{Macros}
Aucune.
\subsubsection{Structures de données}
Aucune.
\subsubsection{Fonctions}
\noindent\hrulefill
\input{common_text_show_warranty_func.tex.tmp}
\input{common_text_show_warranty_com.tex.tmp}
\input{common_text_show_help_func.tex.tmp}
\input{common_text_show_help_com.tex.tmp}
\subsection{Fichier {\texttt{common\_tooltip}}}
Ce fichier génère les tooltips dans l'interface GTK+3.
\subsubsection{Macros}
Aucune.
\subsubsection{Structures de données}
Aucune.
\subsubsection{Fonctions}
\noindent\hrulefill
\input{common_tooltip_generation_func.tex.tmp}
\input{common_tooltip_generation_com.tex.tmp}
\section{Module {\texttt{Éléments finis}}}
\subsection{Fichier {\texttt{EF\_appui}}}
Ce fichier permet de gérer les différents types d'appuis définis par l'utilisateur.\par
Lors de la description des algorithmes des calculs éléments finis, un certain nombre de notations mathématiques et de symboles seront utilisées. Leur signification est la suivante :
\begin{itemize}
\item $[.]$ : matrice,
\item \{.\} : vecteur,
\item $[.]^T$ : matrice transposée,
\item $[R]$ : matrice de rotation,
\item $[K]$ : matrice de rigidité du système,
\item $[Ke]$ : matrice de rigidité élémentaire,
\item $\{\Delta\}$ : vecteur des déplacements,
\item $\{F\}$ : vecteur des efforts.
\end{itemize}
\subsubsection{Macros}
Aucune.
\subsubsection{Structures de données}
\input{EF_appuis_Type_EF_Appui_struct.tex.tmp}
\input{EF_appuis_EF_Appui_struct.tex.tmp}
\subsubsection{Fonctions}
\input{EF_appuis_init_func.tex.tmp}
\input{EF_appuis_init_com.tex.tmp}
\input{EF_appuis_ajout_func.tex.tmp}
\input{EF_appuis_ajout_com.tex.tmp}
\input{EF_appuis_cherche_numero_func.tex.tmp}
\input{EF_appuis_cherche_numero_com.tex.tmp}
\input{EF_appuis_free_func.tex.tmp}
\input{EF_appuis_free_com.tex.tmp}
\subsection{Fichier {\texttt{EF\_charge\_noeud}}}
Ce fichier gère les charges ponctuelles appliquées aux noeuds de la structure.
\subsubsection{Macros}
Aucune.
\subsubsection{Structures de données}
\input{EF_charge_noeud_Charge_Type_struct.tex.tmp}
\input{EF_charge_noeud_Charge_Noeud_struct.tex.tmp}
\subsubsection{Fonctions}
\input{EF_charge_noeud_ajout_func.tex.tmp}
\input{EF_charge_noeud_ajout_com.tex.tmp}
\subsection{Fichier {\texttt{EF\_charge\_barre\_ponctuelle}}}
Ce fichier gère les charges ponctuelles appliquées sur une barre à une position quelconque.
\subsubsection{Macros}
Aucune.
\subsubsection{Structures de données}
\input{EF_charge_barre_ponctuelle_Charge_Barre_Ponctuelle_struct.tex.tmp}
\subsubsection{Fonctions}
\input{EF_charge_barre_ponctuelle_ajout_func.tex.tmp}
\input{EF_charge_barre_ponctuelle_ajout_com.tex.tmp}
\subsection{Fichier {\texttt{EF\_calculs}}}
Ce fichier permet de réaliser tous les calculs éléments finis tel que la factorisation de la matrice de rigidité globale ou encore la détermination des déplacements des noeuds et la courbe des sollicitations dans les barres.
\subsubsection{Macros}
Aucune.
\subsubsection{Structures de données}
Aucune.
\subsubsection{Fonctions}
\input{EF_calculs_initialise_func.tex.tmp}
\input{EF_calculs_initialise_com.tex.tmp}
\input{EF_calculs_genere_mat_rig_func.tex.tmp}
\input{EF_calculs_genere_mat_rig_com.tex.tmp}
\input{EF_calculs_resoud_charge_func.tex.tmp}
\input{EF_calculs_resoud_charge_com.tex.tmp}
\input{EF_calculs_affiche_resultats_func.tex.tmp}
\input{EF_calculs_affiche_resultats_com.tex.tmp}
\subsection{Fichier {\texttt{EF\_noeud}}}
Ce fichier permet gérer les noeuds.
\subsubsection{Macros}
Aucune.
\subsubsection{Structures de données}
\input{EF_noeud_Type_EF_Point_struct.tex.tmp}
\input{EF_noeud_Type_EF_Noeud_struct.tex.tmp}
\subsubsection{Fonctions}
\input{EF_noeuds_init_func.tex.tmp}
\input{EF_noeuds_init_com.tex.tmp}
\input{EF_noeuds_ajout_func.tex.tmp}
\input{EF_noeuds_ajout_com.tex.tmp}
\input{EF_noeuds_cherche_numero_func.tex.tmp}
\input{EF_noeuds_cherche_numero_com.tex.tmp}
\input{EF_noeuds_free_func.tex.tmp}
\input{EF_noeuds_free_com.tex.tmp}
\subsection{Fichier {\texttt{EF\_relachement}}}
Ce fichier permet gérer la liste des relachements.
\subsubsection{Macros}
Aucune.
\subsubsection{Structures de données}
\input{EF_relachement_Type_EF_Relachement_struct.tex.tmp}
\input{EF_relachement_EF_Relachement_struct.tex.tmp}
\subsubsection{Fonctions}
\input{EF_relachement_init_func.tex.tmp}
\input{EF_relachement_init_com.tex.tmp}
\input{EF_relachement_ajout_func.tex.tmp}
\input{EF_relachement_ajout_com.tex.tmp}
\input{EF_relachement_cherche_numero_func.tex.tmp}
\input{EF_relachement_cherche_numero_com.tex.tmp}
\input{EF_relachement_free_func.tex.tmp}
\input{EF_relachement_free_com.tex.tmp}
\subsection{Fichier {\texttt{EF\_rigidite}}}
Ce fichier permet gérer l'initialisation à 0 des matrices de rigidité et leur libération.
\subsubsection{Macros}
Aucune.
\subsubsection{Structures de données}
Aucune.
\subsubsection{Fonctions}
\input{EF_rigidite_init_func.tex.tmp}
\input{EF_rigidite_init_com.tex.tmp}
\input{EF_rigidite_free_func.tex.tmp}
\input{EF_rigidite_free_com.tex.tmp}
\section{Module norme {\texttt{1990}}}
\subsection{Fichier {\texttt{1990\_actions}}}
Ce fichier permet gérer les actions, conformément à l'Eurocode 0.\par
Le type d'une action correspond, par exemple, à la neige en Finlande, à une charge d'exploitation de type stockage. C'est à partir du type que sera déterminés les coefficients définissant les valeurs caractéristiques, fréquentes et quasi-permanentes, conformément au tableau A1.1 de l'Eurocode 0.\par
La catégorie d'une action correspond, par exemple, au poids propre du bâtiment, aux charges variables (vent, neige ou charge d'exploitation sans distinction) ou encore aux charges accidentelles. C'est à partir de la catégorie que sera déterminés les coefficients partiels de sécurité, conformément à la section 6 de l'Eurocode 0.
\subsubsection{Macros}
Aucune.
\subsubsection{Structures de données}
\input{1990_actions_Action_Categorie_struct.tex.tmp}
\input{1990_actions_Action_struct.tex.tmp}
\subsubsection{Fonctions}
\input{_1990_action_type_bat_txt_eu_func.tex.tmp}
\input{_1990_action_type_bat_txt_eu_com.tex.tmp}
\input{_1990_action_type_bat_txt_fr_func.tex.tmp}
\input{_1990_action_type_bat_txt_fr_com.tex.tmp}
\input{_1990_action_type_bat_txt_func.tex.tmp}
\input{_1990_action_type_bat_txt_com.tex.tmp}
\input{_1990_action_categorie_bat_eu_func.tex.tmp}
\input{_1990_action_categorie_bat_eu_com.tex.tmp}
\input{_1990_action_categorie_bat_fr_func.tex.tmp}
\input{_1990_action_categorie_bat_fr_com.tex.tmp}
\input{_1990_action_categorie_bat_func.tex.tmp}
\input{_1990_action_categorie_bat_com.tex.tmp}
\input{_1990_action_init_func.tex.tmp}
\input{_1990_action_init_com.tex.tmp}
\input{_1990_action_ajout_func.tex.tmp}
\input{_1990_action_ajout_com.tex.tmp}
\input{_1990_action_cherche_numero_func.tex.tmp}
\input{_1990_action_cherche_numero_com.tex.tmp}
\input{_1990_action_affiche_tout_func.tex.tmp}
\input{_1990_action_affiche_tout_com.tex.tmp}
\input{_1990_action_free_func.tex.tmp}
\input{_1990_action_free_com.tex.tmp}
\subsection{Fichier {\texttt{1990\_coef\_psi}}}
Ce fichier permet gérer les coefficients psi définis par l'Eurocode 0 et nécessaire pour les combinaisons.
\subsubsection{Macros}
Aucune.
\subsubsection{Structures de données}
Aucune.
\subsubsection{Fonctions}
\input{_1990_coef_psi0_bat_eu_func.tex.tmp}
\input{_1990_coef_psi0_bat_eu_com.tex.tmp}
\input{_1990_coef_psi1_bat_eu_func.tex.tmp}
\input{_1990_coef_psi1_bat_eu_com.tex.tmp}
\input{_1990_coef_psi2_bat_eu_func.tex.tmp}
\input{_1990_coef_psi2_bat_eu_com.tex.tmp}
\input{_1990_coef_psi0_bat_fr_func.tex.tmp}
\input{_1990_coef_psi0_bat_fr_com.tex.tmp}
\input{_1990_coef_psi1_bat_fr_func.tex.tmp}
\input{_1990_coef_psi1_bat_fr_com.tex.tmp}
\input{_1990_coef_psi2_bat_fr_func.tex.tmp}
\input{_1990_coef_psi2_bat_fr_com.tex.tmp}
\input{_1990_coef_psi0_bat_func.tex.tmp}
\input{_1990_coef_psi0_bat_com.tex.tmp}
\input{_1990_coef_psi1_bat_func.tex.tmp}
\input{_1990_coef_psi1_bat_com.tex.tmp}
\input{_1990_coef_psi2_bat_func.tex.tmp}
\input{_1990_coef_psi2_bat_com.tex.tmp}
\subsection{Fichier {\texttt{1990\_combinaisons}}}
Ce fichier permet gérer de générer les combinaisons conformément à celles définies dans l'Eurocode 0.
\subsubsection{Macros}
Aucune.
\subsubsection{Structures de données}
\input{1990_combinaisons_Combinaison_Element_struct.tex.tmp}
\input{1990_combinaisons_Combinaison_struct.tex.tmp}
\input{1990_combinaisons_Combinaisons_struct.tex.tmp}
\subsubsection{Fonctions}
\input{_1990_combinaisons_genere_xor_func.tex.tmp}
\input{_1990_combinaisons_genere_xor_com.tex.tmp}
\input{_1990_combinaisons_genere_and_func.tex.tmp}
\input{_1990_combinaisons_genere_and_com.tex.tmp}
\input{_1990_combinaisons_genere_or_func.tex.tmp}
\input{_1990_combinaisons_genere_or_com.tex.tmp}
\input{_1990_combinaisons_init_func.tex.tmp}
\input{_1990_combinaisons_init_com.tex.tmp}
\input{_1990_combinaisons_genere_func.tex.tmp}
\input{_1990_combinaisons_genere_com.tex.tmp}
\input{_1990_combinaisons_free_func.tex.tmp}
\input{_1990_combinaisons_free_com.tex.tmp}
\subsection{Fichier {\texttt{1990\_duree}}}
Ce fichier permet de gérer les durées indicatives d'un projet conformément à l'Eurocode 0.
\subsubsection{Macros}
Aucune.
\subsubsection{Structures de données}
Aucune.
\subsubsection{Fonctions}
\input{_1990_duree_projet_eu_func.tex.tmp}
\input{_1990_duree_projet_eu_com.tex.tmp}
\input{_1990_duree_projet_fr_func.tex.tmp}
\input{_1990_duree_projet_fr_com.tex.tmp}
\input{_1990_duree_projet_func.tex.tmp}
\input{_1990_duree_projet_com.tex.tmp}
\input{_1990_duree_projet_txt_eu_func.tex.tmp}
\input{_1990_duree_projet_txt_eu_com.tex.tmp}
\input{_1990_duree_projet_txt_fr_func.tex.tmp}
\input{_1990_duree_projet_txt_fr_com.tex.tmp}
\input{_1990_duree_projet_txt_func.tex.tmp}
\input{_1990_duree_projet_txt_com.tex.tmp}
\subsection{Fichier {\texttt{1990\_groupes}}}
Ce fichier permet gérer les groupes de combinaisons.
\subsubsection{Macros}
Aucune.
\subsubsection{Structures de données}
\input{1990_groupes_Element.tex.tmp}
\input{1990_groupes_Type_Groupe_Combinaison.tex.tmp}
\input{1990_groupes_Groupe.tex.tmp}
\input{1990_groupes_Niveau_Groupe.tex.tmp}
\subsubsection{Fonctions}
\input{_1990_groupe_init_func.tex.tmp}
\input{_1990_groupe_init_com.tex.tmp}
\input{_1990_groupe_ajout_niveau_func.tex.tmp}
\input{_1990_groupe_ajout_niveau_com.tex.tmp}
\input{_1990_groupe_ajout_groupe_func.tex.tmp}
\input{_1990_groupe_ajout_groupe_com.tex.tmp}
\input{_1990_groupe_ajout_element_func.tex.tmp}
\input{_1990_groupe_ajout_element_com.tex.tmp}
\input{_1990_groupe_positionne_niveau_func.tex.tmp}
\input{_1990_groupe_positionne_niveau_com.tex.tmp}
\input{_1990_groupe_positionne_groupe_func.tex.tmp}
\input{_1990_groupe_positionne_groupe_com.tex.tmp}
\input{_1990_groupe_positionne_element_func.tex.tmp}
\input{_1990_groupe_positionne_element_com.tex.tmp}
\input{_1990_groupe_free_niveau_func.tex.tmp}
\input{_1990_groupe_free_niveau_com.tex.tmp}
\input{_1990_groupe_free_groupe_func.tex.tmp}
\input{_1990_groupe_free_groupe_com.tex.tmp}
\input{_1990_groupe_free_element_func.tex.tmp}
\input{_1990_groupe_free_element_com.tex.tmp}
\input{_1990_groupe_free_func.tex.tmp}
\input{_1990_groupe_free_com.tex.tmp}
\input{_1990_groupe_affiche_tout_func.tex.tmp}
\input{_1990_groupe_affiche_tout_com.tex.tmp}
\subsection{Fichier {\texttt{1990\_gtk\_groupes}}}
Ce fichier permet gérer la création et la gestion des groupes ainsi que la génération des combinaisons et des pondérations par l'utilisation d'une interface gtk+3.
\subsubsection{Macros}
Aucune.
\subsubsection{Structures de données}
Aucune externe.
\subsubsection{Fonctions}
\input{_1990_gtk_groupes_func.tex.tmp}
\input{_1990_gtk_groupes_com.tex.tmp}
\subsection{Fichier {\texttt{1990\_ponderations}}}
Ce fichier permet gérer les pondérations déduites des combinaisons.
\subsubsection{Macros}
Aucune.
\subsubsection{Structures de données}
\input{1990_ponderations_Ponderation_Element.tex.tmp}
\input{1990_ponderations_Ponderation.tex.tmp}
\subsubsection{Fonctions}
\input{_1990_ponderations_genere_un_func.tex.tmp}
\input{_1990_ponderations_genere_un_com.tex.tmp}
\input{_1990_ponderations_genere_eu_func.tex.tmp}
\input{_1990_ponderations_genere_eu_com.tex.tmp}
\input{_1990_ponderations_genere_fr_func.tex.tmp}
\input{_1990_ponderations_genere_fr_com.tex.tmp}
\input{_1990_ponderations_genere_func.tex.tmp}
\input{_1990_ponderations_genere_com.tex.tmp}
\input{_1990_ponderations_affiche_tout_func.tex.tmp}
\input{_1990_ponderations_affiche_tout_com.tex.tmp}
\section{Module norme {\texttt{1992\_1\_1}}}
\subsection{Fichier {\texttt{1992\_1\_1\_barres}}}
Ce fichier permet gérer les barres en béton conformément à l'Eurocode 2, partie 1-1.
\subsubsection{Macros}
Aucune.
\subsubsection{Structures de données}
\input{1992_1_1_barres_Beton_Barre.tex.tmp}
\subsubsection{Fonctions}
\input{_1992_1_1_barres_init_func.tex.tmp}
\input{_1992_1_1_barres_init_com.tex.tmp}
\input{_1992_1_1_barres_ajout_func.tex.tmp}
\input{_1992_1_1_barres_ajout_com.tex.tmp}
\input{_1992_1_1_barres_cherche_numero_func.tex.tmp}
\input{_1992_1_1_barres_cherche_numero_com.tex.tmp}
\input{_1992_1_1_barres_rigidite_ajout_func.tex.tmp}
\input{_1992_1_1_barres_rigidite_ajout_com.tex.tmp}
\input{_1992_1_1_barres_rigidite_ajout_tout_func.tex.tmp}
\input{_1992_1_1_barres_rigidite_ajout_tout_com.tex.tmp}
\input{_1992_1_1_barres_free_func.tex.tmp}
\input{_1992_1_1_barres_free_com.tex.tmp}
\subsection{Fichier {\texttt{1992\_1\_1\_materiaux}}}
Ce fichier permet gérer les matériaux en béton.
\subsubsection{Macros}
Aucune.
\subsubsection{Structures de données}
\input{1992_1_1_materiaux_Beton_Materiau_struct.tex.tmp}
\subsubsection{Fonctions}
\input{_1992_1_1_materiaux_init_func.tex.tmp}
\input{_1992_1_1_materiaux_init_com.tex.tmp}
\input{_1992_1_1_materiaux_ajout_func.tex.tmp}
\input{_1992_1_1_materiaux_ajout_com.tex.tmp}
\input{_1992_1_1_materiaux_cherche_numero_func.tex.tmp}
\input{_1992_1_1_materiaux_cherche_numero_com.tex.tmp}
\input{_1992_1_1_materiaux_free_func.tex.tmp}
\input{_1992_1_1_materiaux_free_com.tex.tmp}
\subsection{Fichier {\texttt{1992\_1\_1\_section}}}
Ce fichier permet gérer les sections en béton.
\subsubsection{Macros}
Aucune.
\subsubsection{Structures de données}
\input{1992_1_1_section_Type_Beton_Section_struct.tex.tmp}
\input{1992_1_1_section_Beton_Section_Caracteristiques_struct.tex.tmp}
\input{1992_1_1_section_Beton_Section_Rectangulaire_struct.tex.tmp}
\input{1992_1_1_section_Beton_Section_T_struct.tex.tmp}
\input{1992_1_1_section_Beton_Section_Carre_struct.tex.tmp}
\input{1992_1_1_section_Beton_Section_Circulaire_struct.tex.tmp}
\subsubsection{Fonctions}
\input{_1992_1_1_sections_init_func.tex.tmp}
\input{_1992_1_1_sections_init_com.tex.tmp}
\input{_1992_1_1_sections_ajout_rectangulaire_func.tex.tmp}
\input{_1992_1_1_sections_ajout_rectangulaire_com.tex.tmp}
\input{_1992_1_1_sections_ajout_T_func.tex.tmp}
\input{_1992_1_1_sections_ajout_T_com.tex.tmp}
\input{_1992_1_1_sections_ajout_carre_func.tex.tmp}
\input{_1992_1_1_sections_ajout_carre_com.tex.tmp}
\input{_1992_1_1_sections_ajout_circulaire_func.tex.tmp}
\input{_1992_1_1_sections_ajout_circulaire_com.tex.tmp}
\input{_1992_1_1_sections_cherche_numero_func.tex.tmp}
\input{_1992_1_1_sections_cherche_numero_com.tex.tmp}
\input{_1992_1_1_sections_free_func.tex.tmp}
\input{_1992_1_1_sections_free_com.tex.tmp}
\end{document}
